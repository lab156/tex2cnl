\documentclass{article}

\usepackage{amsmath,amsfonts}
\usepackage{tex2cnl}

\newcommand{\RR}{\mathbb{R}}
\newcommand{\pder}[2][]{\frac{\partial#1}{\partial#2}}
\newcommand{\ellipMacro}[1]{x^{#1}}
\newcommand{\otherMacro}[1]{\alpha^{#1}}

%\listadd\csList{CNLRR}%

\begin{document}
\noindent
First we define some macros to test \verb|\CNLexpand| commands:\\
The macro \verb|\RR| expands to:  $$\RR $$.\\
The macro \verb|$$\pder{x}$$| expands:  $$\pder{x}$$
The macro \verb|$$\pder[f]{x}$$| expands:  $$\pder[f]{x}$$

\noindent
Then we specify the behaviour of some macros by including: 
\begin{verbatim}
\CNLnoexpand\RR
\CNLdelete\pder
\CNLdelete[1]\nabla
\CNLdelete[2]\frac
\end{verbatim}

\CNLnoexpand\RR
\CNLdelete\pder
\CNLdelete[1]\nabla
\CNLexpand\int
\CNLdelete[2]\frac


\noindent
The list of all commands that have been ``CNLed'' is available in a list:\\
Optional arguments are added so the table can be displayed

\begin{tabular}{|c|c|c|}
    \hline
    \textbf{Supported Command} & \textbf{Expansion} &\textbf{Expand behaviour} \\
    \hline
    \printList
    \hline
\end{tabular}

\noindent
and $$\ellipMacro{x}$$
How about the $$\frac{1}{2}$$

We have to demo ellip inline: $\ellip{ellipMacro}{1}{10}{+}$ 
$$\ellip{ellipMacro}{1}{n}{+} = (x^{n+1}-1)/(x-1)$$ 
Now with other macro: $\ellip{otherMacro}{1}{5}{-}$
\end{document}
